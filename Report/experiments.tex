\section{Experiments}
We have performed several experiments on our system, both to show that our security measures work, to find optimal parameters and to compare flooding and k-walker approaches with our added security features.

\subsection{Proof-of-Work hinders Sybil Attacks}
(TODO: ROLAND)

\subsection{Finding optimal k-walker parameters}
As described in section~\ref{subsec:base_system}, the k-walker algorithm implements an eventual delivery guarantee as long as sender remains connected to the network by sending out walkers with exponentially increasing TTL at exponentially increasing intervals. Before doing the experiments comparing flooding and k-walkers we needed to find some optimal or at least pretty good parameters for this.

The required parameters were the starting values of wait time and TTL as well as how many walkers to send in parallel each time. Tested values of starting wait time include 0.5, 1, 2, and tested starting values of TTL include 8, 16, 32, 64, 128 and the tested numbers of walkers to send out time include 1, 2, 4, 8, 16. Since each test took quite a while and had to be repeated several times to try different network layouts, we didn't test all possible combinations of the values but used our intuition and understanding of the effect of the values to zero in on some good values to test further. We found that the network layout had a huge effect on the time and number of passed messages in the network to send a single message and receive verification of delivery. We omit presenting the data from this experiments as not even close to every combination of the parameters were tested, making graphing the values useless, and not every test was written down.


\subsection{Flooding vs. k-walkers}
We tested the time and internal messages passed in the whole network by having two peers join a network of 100 peers and sending 10 messages from one to another, waiting for the acknowledgement of the previous message before sending the next. This test was then repeated to test various network layouts. These tests were performed with Proof-of-Work turned off and no artificial latency introduced in the system.

Sadly we encountered a bug or a limitation of our testing machine that meant the flooding algorithm didn't reliably for for networks much above 100 peers. The verification simply did not return reliably through the network to the sender, causing the test to pause infinitely and making us restrict our testing to networks less than 100 peers in size. This is unfortunate as we expect the k-walker algorithm to only outperform the flooding algorithm at higher numbers of peers. We will instead try to show that it is plausible that the k-walker algorithm outperforms the flooding algorithm in bigger networks by showing that the performance of the k-walker algorithm relative to the flooding algorithm goes up as the network size increases.

(TODO: Fejlen er fixet. 2 tests, 1 der sammenligner beskeder, og 1 der sammenligner latency)

\subsection{Cardinality of Diffie-Hellman Pools}

When working with the Pooled Diffie-Hellman connections, we wished to ensure that we made as few Diffie-Hellman negotions as possible. In order to measure this performance parameter, we started a network, and give it time to settle down, while performing cover traffic. We then measure the average cardinality of the connection pools residing in the peers. This number should preferably be close to 1.

The test was done specifically with 100ms faked latency between peers, and by waiting 2 minutes while cover traffic and network kept generating data. We ran the test for 10 to 100 peers, in increments of 10. A higher number of peers was deemed unnecessary, as peers seemed to have little influence on the cardinality of the connection pools.

(TODO: Indsæt Graf, den er i Grapes)

From this test, we see that we have a cardinality ranging between 1.07 and 1.25, which means that we are generating 7-25\% additional additional connections. The average amount of connection overshooting is very close to 12\% which we deem to be good. Note that the cardinality does not appear rise with an increasing number of peers, and in fact the highest numbers seem to reside at very low peer numbers. We theorise that the high cardinality of low peer numbers might be caused by a greater amount of loops in the RPC call graph, leading to a forced additional connection.