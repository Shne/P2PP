\section{Further Work}

Throughout this project, we have managed to meet our baseline goals for the system, and make a working command line chat client that is heavily secured. However, there is still much to be done before the system is enterprise-strength ready.
Most importantly is a proper user interface. The current command line interface is not entirely intuitive, and lacks proper conversation threading.

Optimized k-walkers, such as Adaptive Probabilistic, could be a huge improvement. Currently our k-walkers are very basic, due to the lack of information about sender, receiver and content. Since we lack a lot of the information usually used to optimize k-walkers, it might prove difficult to make them perform better than they currently do, but it would be worth a try. Some of the available optimizations for k-walkers might also compromise the security of the system by introducing exploitable behaviour of the k-walkers.

A better cover traffic scheme could be implemented. The current cover traffic is completely random, and this leads to a slight possibility for traffic analysis against very active peers. Using constant-rate cover traffic would be preferable.
In order to avoid eclipse attacks, it would be quite a good idea to avoid peers from the same subnet. Such a thing could be easily  done by denying neighbour request from peers in an  already connected subnet.

It would also be interesting to be able to send files using our system. But then if the file that was sent was very large we would need some mechanism to lower the amount of cover traffic to avoid putting to high a strain on the network.

Another functionality we could add could be group chat. This would make it easier for peers to share the same information with a group of other peers at the same time. 

These improvements are only a few of many, there are countless of small changes that would make the system both safer, and more user friendly.