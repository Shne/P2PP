\section{Manual of Operations}

In order to use the project, you will need \texttt{Python 3.3} with the \texttt{PyCrypto} Module installed.

Included in the pre-package solution are the following files;
\begin{description}
\item[peer.py] The base peer class, this provides the logic for the peers of the network.
\item[interactive\_peer.py] The command-line interface for the peer.
\item[setup.py] A script generating a large amount of peers.
\item[hashcash.py] The hashcash dependency \citep{HashCash}.
\item[public.pem, private.pem, public2.pem, private2.pem] Sample 2048-bit RSA keys.
\item[DH.pem] Precomputed Diffie-Hellman data. 
\end{description}

In order to start a peer, use the \texttt{interactive\_peer.py} python file.



\subsection{Operating the Command-Line Interface}
In order to operate a peer in the network, one must rely on either python scripting against the peer class of the source code, or the supplied command line interface.

We here explain the commands required to operate the peer using the command line interface. Scripting directly against peer class is left as an exercise for the reader.

Note that the peer still supports most of the commands of the original network \citep{P2PN}.

The commands are as follows:

\begin{description}
\item[hello [\textit{address}]] Attempt to join the network. An optional address parameter may be specified in order to bootstrap against a known peer.

After joining the network, the peer will be ready to add keys, and chat. Please note that it might take several seconds for the peer to establish an acceptable amount of neighbours.

\item[secret \textit{private\_key}]
Load the given private key from a local file, and set is as the current key used for decrypting and signing messages. 
This is required in order to receive messages encrypted with the corresponding public key, and to sign messages sent from the local peer.
Note that the key must be a RSA private key in the \textit{pem} format.

\item[friend \textit{name public\_key}]
Load the given public key from a local file, and associate it with the alias provided by the name parameter. 
This is required in order to send messages to the peer with the corresponding private key, and to identify that peer as a sender.
Note that the key must be a RSA public key in the \textit{pem} format.

\item[publish \textit{public\_key}]
Make the given public key available for retrieval through the peer to peer network. Shortly after entering this command the peer will display a hash of your key, which you can share. This allows other peers to download your public key through the network if they have the corresponding hash string.
Note that the key must be a RSA public key in the \textit{pem} format.

\item[friend \textit{name hash}]
Fetches the key stored in the network under the given hash, and checks for hash validity of the key. If successful, the public key retrieved will be associated with the alias specified in the name parameter.

\item[message \textit{name message}]
Attempts to deliver a message to a friend added under the alias specified by the name parameter, with the content of the message parameter, using flooding.
The message will be signed if possible, and a report of delivery given.

\item[kmessage \textit{name message}]
Attempts to deliver a message to a friend added under the alias specified by the name parameter, with the content of the message parameter, using k-walkers.
The message will be signed if possible, and an id will be given, which allows matching to a later received acknowledgement.

\end{description}